The basic idea of the Chameleon-\/\-Mini was to integrate so-\/called configurations to define the device's behaviour. You can choose the active configuration via the command-\/line using the {\ttfamily C\-O\-N\-F\-I\-G} command (See \hyperlink{Page_CommandLine}{The chameleon command line}).

Below is a list of the configurations with a brief description of what it is and how it works. In configurations where an actual card is simulated, the card memory has to be uploaded to the Chameleon before accessing it by a reader. This is done using the X-\/\-M\-O\-D\-E\-M protocol on the command-\/line (See \hyperlink{Page_CommandLine}{The chameleon command line}).

\section*{Configurations }

\subsection*{{\ttfamily N\-O\-N\-E} }

The Chameleon's R\-F interface is not activated and it behaves entirely passive.

\subsection*{{\ttfamily M\-F\-\_\-\-U\-L\-T\-R\-A\-L\-I\-G\-H\-T} / {\ttfamily M\-F\-\_\-\-C\-L\-A\-S\-S\-I\-C\-\_\-1\-K} / {\ttfamily M\-F\-\_\-\-C\-L\-A\-S\-S\-I\-C\-\_\-4\-K} }

The Chameleon behaves like a Mifare Ultralight or Classic (with 1\-K or 4\-K E\-E\-P\-R\-O\-M).

The memory layout of these configurations is as shown in the datasheets from N\-X\-P and thus fully compatible with the dumpfiles from libnfc.

\subsection*{{\ttfamily I\-S\-O14443\-A\-\_\-\-S\-N\-I\-F\-F} / {\ttfamily I\-S\-O15693\-\_\-\-S\-N\-I\-F\-F} }

These are sniffing configurations, where no answers to any requests from a reader are done. Thus the Chameleon behaves completely passive, which is useful if a second card is present in the reader field. Then this configuration can be used in conjunction with one of the logging modes (See \hyperlink{Page_Logging}{Logging capabilities}) to sniff for data, e.\-g. the U\-I\-D that is sent from the reader to the secondary card.

Note that especially with the I\-S\-O14443\-A sniffing mode, it may happen that the load modulation of the secondary card trips the codec and thus logs some garbled data. 